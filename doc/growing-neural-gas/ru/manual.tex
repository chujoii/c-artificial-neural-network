\documentclass[unicode, 12pt, a4paper,oneside,fleqn]{article}

\input{../../../TeX/preambula-ru.tex}
\usepackage[colorlinks=true]{hyperref} % url hyperlink (beamer already include it, so move here for prevent conflict)

\author{Роман В. Приходченко}

\title{Руководство по работе с программой}


\makeindex



\begin{document}

% меняем английские термины на русские
\renewcommand\bibname{СПИСОК ЛИТЕРАТУРЫ}
\renewcommand\refname{\centering Список литературы}
\renewcommand\contentsname{\centering Содержание}


% образцы переноса сложных слов - не работает?
% \hyphenation{веб=-ин-тер-фей-се веб-ин-тер-фей-с}
% or use in text: веб"=интерфейс (require: \usepackage[russian]{babel})

% печатаем титульный лист
\makeatletter % generate \@title, \@date, ...
\maketitle

\begin{table}[ht]
  \begin{tabular}{cc}
    \includegraphics[width=2cm]{../../../CC_BY-SA_88x31.png} &
    \shortstack{руководство распространяется в соответствии с
      условиями\\
      \href{http://creativecommons.org/licenses/by-sa/3.0/}{Attribution-ShareAlike} \\
      (Атрибуция — С сохранением условий) CC BY-SA \\
      Копирование и распространение приветствуется.}
  \end{tabular}
\end{table}

\newpage
% печатаем оглавление
\tableofcontents

\section{Алгоритм}
Описание алгоритма и список основных источников в файле algorithm.pdf


\clearpage
\section{Что содержит внутри себя нейрон и база знаний?}

Юнит-тесты и примеры работы нейронных сетей лежат в каталоге
\verb'c-artificial-neural-network/src/example/growing-neural-gas/static-data'
\begin{itemize}
\item \verb'unit-testing.c' --- проверка работы всех функций (можно
  пройтись по всему алгоритму)
\item \verb'gng-to-dot_unit-testing.c' --- проверка работы функций
  вывода в DOT файл (pq.gv)
\item \verb'example.c' --- пример работы маленькой нейронной сети
\end{itemize}

В файле \verb'growing-neural-gas.h' описана структура нейрона,
заметьте которой есть ссылки на массивы \verb'weight' и
\verb'conn_age':

\begin{lstlisting}[language=C,
  caption={Описание структуры нейрона},
  captionpos=b,
  label={nueron:code}]
typedef struct Neuron {
	unsigned char active;
        float *weight; // DIMENSION_OF_SENSOR * sizeof (* (gng[i].weight))
	int *conn_age; // LIMIT_NETWORK_SIZE * sizeof (* (gng[i].conn_age))
        float local_error;
	float utility_factor;
	int group; // only for visualization
} NEURON;
\end{lstlisting}


Нейронная сеть это просто массив нейронов (пример из файлов:
\verb'unit-testing.c', \verb'gng-to-dot_unit-testing.c',
\verb'example.c', \verb'clusterizator.c', ).  Здесь выделяется место
для массива нейронов.  Для каждого нейрона дополнительно выделяется
память для массивов \verb'weight' и \verb'conn_age':

\begin{lstlisting}[language=C,
  caption={Выделение памяти},
  captionpos=b,
  label={malloc:code}]
NEURON *pq_gng = malloc(LIMIT_NETWORK_SIZE * sizeof(* pq_gng));
if (pq_gng == NULL) {
	return 1;
}
for (int i=0; i<LIMIT_NETWORK_SIZE; i++) {
	pq_gng[i].weight = malloc(DIMENSION_OF_SENSOR * sizeof (* (pq_gng[i].weight)));
	if (pq_gng[i].weight == NULL) {
		// fixme: free (pq_gng); and already allocation weights and conn_ages
		return 2;
	}
	pq_gng[i].conn_age = malloc(LIMIT_NETWORK_SIZE * sizeof (* (pq_gng[i].conn_age)));
	if (pq_gng[i].conn_age == NULL) {
		// fixme: free (pq_gng); and already allocation weights and conn_ages
		return 2;
	}
}
\end{lstlisting}

И конечно же требуется освободить память в конце работы:
\begin{lstlisting}[language=C,
  caption={Освобождение памяти},
  captionpos=b,
  label={free:code}]
/* free memory */
for (int i=0; i<LIMIT_NETWORK_SIZE; i++) {
	free (pq_gng[i].weight);
	free (pq_gng[i].conn_age);
}
free (pq_gng);
\end{lstlisting}



\clearpage
\section{Как можно прочитать информацию, которую содержит нейрон и база знаний?}

В файле \verb'growing-neural-gas.c' есть функция для отладочной печати
\verb'print_neuron'.  Её можно добавить в файл \verb'clusterizator.c',
но предварительно необходимо уменьшить размер нейронной сети в файле
\verb'clusterizator.h' (иначе нейронная сеть просто не влезет в экран)

\begin{lstlisting}[language=C,
  caption={Файл clusterizator.h},
  captionpos=b,
  label={free:code}]
#define LIMIT_NETWORK_SIZE 10
\end{lstlisting}

Также на невлезание в экран --- влияет размер
\verb'DIMENSION_OF_SENSOR', поэтому печать весов (сейчас 1128) можно
отключить в \verb'growing-neural-gas.c' закомментировав
соответствующие строки в функции \verb'print_neuron'.



Следующие добавки могут испортить вывод номера эпохи в
\verb'epoch.txt', и соответственно изменить режим обучения (решение:
использовать \verb'fprintf(stderr, ...)' вместо \verb'printf'). Чтобы
добавить в файл \verb'clusterizator.c' отладочную печать:
\begin{enumerate}
\item после считывания из файла базы знаний (содержимое
  knowledge-base.gng) и
\item после расчёта по алгоритму растущего нейронного газа
  (соответственно перед записью в файл \verb'knowledge-base.gng')
\end{enumerate}
В функции main необходимо добавить строки:

\begin{lstlisting}[language=C,
  caption={Файл clusterizator.c},
  captionpos=b,
  label={free:code}]
if ( access(KNOWLEDGE_BASE_GNG_FILENAME, F_OK) != -1 ) { // file exists
	read_gng_from_file (KNOWLEDGE_BASE_GNG_FILENAME, DIMENSION_OF_SENSOR, LIMIT_NETWORK_SIZE, pq_gng);
+	for (int i=0; i<LIMIT_NETWORK_SIZE; i++) {
+		print_neuron (DIMENSION_OF_SENSOR, LIMIT_NETWORK_SIZE, pq_gng[i]);
+        }
	if (epoch == 0L) {
//...........................
extract_groups_from_conn_ages (LIMIT_NETWORK_SIZE, pq_gng);
+for (int i=0; i<LIMIT_NETWORK_SIZE; i++) {
+	print_neuron (DIMENSION_OF_SENSOR, LIMIT_NETWORK_SIZE, pq_gng[i]);
+}
write_gng_to_file (KNOWLEDGE_BASE_GNG_FILENAME, DIMENSION_OF_SENSOR, LIMIT_NETWORK_SIZE, pq_gng);
\end{lstlisting}

\end{document}
