\documentclass[unicode, 12pt, a4paper,oneside,fleqn]{article}

\input{../../../TeX/preambula-ru.tex}
\usepackage[colorlinks=true]{hyperref} % url hyperlink (beamer already include it, so move here for prevent conflict)

\author{Роман В. Приходченко}

\title{Руководство по работе с функциями программ и библиотеки
  <<c-artificial-neural-network>>}


\makeindex



\begin{document}

% меняем английские термины на русские
\renewcommand\bibname{СПИСОК ЛИТЕРАТУРЫ}
\renewcommand\refname{\centering Список литературы}
\renewcommand\contentsname{\centering Содержание}


% образцы переноса сложных слов - не работает?
% \hyphenation{веб=-ин-тер-фей-се веб-ин-тер-фей-с}
% or use in text: веб"=интерфейс (require: \usepackage[russian]{babel})

% печатаем титульный лист
\makeatletter % generate \@title, \@date, ...
\maketitle

\begin{table}[ht]
  \begin{tabular}{cc}
    \includegraphics[width=2cm]{../../../CC_BY-SA_88x31.png} &
    \shortstack{руководство распространяется в соответствии с
      условиями\\
      \href{http://creativecommons.org/licenses/by-sa/3.0/}{Attribution-ShareAlike} \\
      (Атрибуция — С сохранением условий) CC BY-SA \\
      Копирование и распространение приветствуется.}
  \end{tabular}
\end{table}

\newpage
% печатаем оглавление
\tableofcontents

\section{Алгоритм}
Описание алгоритма и список основных источников в файле\\
\verb'c-artificial-neural-network/doc/growing-neural-gas/ru/algorithm.pdf'


\clearpage
\section{Что содержит внутри себя нейрон и база знаний?}

Юнит-тесты и примеры работы нейронных сетей лежат в каталоге\\
\verb'c-artificial-neural-network/src/example/growing-neural-gas/static-data'
\begin{itemize}
\item \verb'unit-testing.c' --- проверка работы всех функций (можно
  пройтись по всему алгоритму). \todo{Необходима доработка: добавить счётчик
  успешно выполненных функций (сейчас только визуальная проверка).}
\item \verb'gng-to-dot_unit-testing.c' --- проверка работы функций
  вывода в DOT файл (pq.gv)
\item \verb'example.c' --- пример работы маленькой нейронной сети
\end{itemize}

В файле \verb'growing-neural-gas.h' описана структура нейрона,
заметьте, структура содержит ссылки на массивы \verb'weight' и
\verb'conn_age':

\begin{lstlisting}[language=C,
  caption={Описание структуры нейрона},
  captionpos=b,
  label={nueron:code}]
typedef struct Neuron {
	unsigned char active;
        float *weight; // DIMENSION_OF_SENSOR * sizeof (* (gng[i].weight))
	int *conn_age; // LIMIT_NETWORK_SIZE * sizeof (* (gng[i].conn_age))
        float local_error;
	float utility_factor;
	int group; // only for visualization
} NEURON;
\end{lstlisting}


Нейронная сеть это просто массив нейронов (пример из файлов:\\
\verb'unit-testing.c', \verb'gng-to-dot_unit-testing.c',
\verb'example.c', \verb'clusterizator.c').  Здесь выделяется место
для массива нейронов.  Для каждого нейрона дополнительно выделяется
память для массивов \verb'weight' и \verb'conn_age':

\begin{lstlisting}[language=C,
  caption={Выделение памяти},
  captionpos=b,
  label={malloc:code},
  linewidth=14cm]
NEURON *pq_gng = malloc(LIMIT_NETWORK_SIZE * sizeof(* pq_gng));
if (pq_gng == NULL) {
	return 1;
}
for (int i=0; i<LIMIT_NETWORK_SIZE; i++) {
	pq_gng[i].weight = malloc(DIMENSION_OF_SENSOR * sizeof (* (pq_gng[i].weight)));
	if (pq_gng[i].weight == NULL) {
		// fixme: free (pq_gng); and already allocation weights and conn_ages
		return 2;
	}
	pq_gng[i].conn_age = malloc(LIMIT_NETWORK_SIZE * sizeof (* (pq_gng[i].conn_age)));
	if (pq_gng[i].conn_age == NULL) {
		// fixme: free (pq_gng); and already allocation weights and conn_ages
		return 2;
	}
}
\end{lstlisting}

И, конечно же, требуется освободить память в конце работы:
\begin{lstlisting}[language=C,
  caption={Освобождение памяти},
  captionpos=b,
  label={free:code}]
/* free memory */
for (int i=0; i<LIMIT_NETWORK_SIZE; i++) {
	free (pq_gng[i].weight);
	free (pq_gng[i].conn_age);
}
free (pq_gng);
\end{lstlisting}



\clearpage
\section{Как можно прочитать информацию, которую содержит нейрон и база знаний?}

В файле \verb'growing-neural-gas.c' есть функция для отладочной печати
\verb'print_neuron'.  Её можно добавить в файл \verb'clusterizator.c',
но предварительно необходимо уменьшить размер нейронной сети в файле\\
\verb'clusterizator.h' (иначе нейронная сеть просто не влезет в экран)

\begin{lstlisting}[language=C,
  caption={Файл clusterizator.h},
  captionpos=b,
  label={network:size:code}]
#define LIMIT_NETWORK_SIZE 10
\end{lstlisting}

Также на невлезание в экран --- влияет размер
\verb'DIMENSION_OF_SENSOR', поэтому печать весов (сейчас 1128) можно
отключить в \verb'growing-neural-gas.c' закомментировав
соответствующие строки в функции \verb'print_neuron'.



Следующие добавки могут испортить вывод номера эпохи в
\verb'epoch.txt', и соответственно изменить режим обучения
(решение: использовать\\
\verb'fprintf(stderr, ...)' вместо \verb'printf').

Чтобы добавить в файл \verb'clusterizator.c' отладочную печать:
\begin{enumerate}
\item после считывания из файла базы знаний (содержимое
  \verb'knowledge-base.gng') и
\item после расчёта по алгоритму растущего нейронного газа
  (соответственно перед записью в файл \verb'knowledge-base.gng')
\end{enumerate}
В функции main необходимо добавить строки:

\begin{lstlisting}[language=C,
  caption={Файл clusterizator.c},
  captionpos=b,
  label={debug:print:code},
  linewidth=16.5cm]
if ( access(KNOWLEDGE_BASE_GNG_FILENAME, F_OK) != -1 ) { // file exists
	read_gng_from_file (KNOWLEDGE_BASE_GNG_FILENAME, DIMENSION_OF_SENSOR, LIMIT_NETWORK_SIZE, pq_gng);
+	for (int i=0; i<LIMIT_NETWORK_SIZE; i++) {
+		print_neuron (DIMENSION_OF_SENSOR, LIMIT_NETWORK_SIZE, pq_gng[i]);
+        }
	if (epoch == 0L) {
//...........................
extract_groups_from_conn_ages (LIMIT_NETWORK_SIZE, pq_gng);
+for (int i=0; i<LIMIT_NETWORK_SIZE; i++) {
+	print_neuron (DIMENSION_OF_SENSOR, LIMIT_NETWORK_SIZE, pq_gng[i]);
+}
write_gng_to_file (KNOWLEDGE_BASE_GNG_FILENAME, DIMENSION_OF_SENSOR, LIMIT_NETWORK_SIZE, pq_gng);
\end{lstlisting}

Вывод отладочной информации добавлен сразу после чтения и перед
записью в базу знаний \verb'knowledge-base.gng' неслучайно. Если
внимательно посмотреть на функции \verb'read_gng_from_file' и
\verb'write_gng_to_file' из Листинга.~\ref{debug:print:code}, то можно
увидеть что данные функции состоят из цикла \verb'for' внутри которого
с помощью \verb'fwrite'/\verb'fread' (на примере записи
Листинга.~\ref{code:write:gng:to:file}):

\begin{lstlisting}[language=C,
  caption={Функция write\_gng\_to\_file из файла growing-neural-gas.c},
  captionpos=b,
  label={code:write:gng:to:file},
  linewidth=16.5cm]
int write_gng_to_file (char *file_name, int dimension_of_sensor, int limit_network_size, NEURON *gng)
...
	for (int i=0; i<limit_network_size; i++) {
		/* fixme: not need to write adress of weight
		 * and conn_age (because it is useless
		 * information for next run of programm) */
		result = fwrite(&(gng[i]), sizeof(struct Neuron), 1, ifp);

		sum_res += (result == 1) ? 1 : 0;
		result = fwrite(gng[i].weight, dimension_of_sensor * sizeof (* (gng[i].weight)), 1, ifp);
		sum_res += (result == 1) ? 1 : 0;
		result = fwrite(gng[i].conn_age, limit_network_size * sizeof (* (gng[i].conn_age)), 1, ifp);
		sum_res += (result == 1) ? 1 : 0;
	}
...
\end{lstlisting}

Таким образом, структура базы знаний хранящаяся в файле\\
\verb'knowledge-base.gng', состоит из следующих элементов:
\begin{itemize}
\item нейрон \verb'gng[i]' (структура нейрона была
  показана в Листинге.~\ref{nueron:code}, выделение памяти в
  Листинге.~\ref{malloc:code})
\item массив весов \verb'gng[i].weight' (ранее было
  показано в виде Листинга.~\ref{malloc:code})
\item массив соединений \verb'gng[i].conn_age' (ранее было показано в виде
  Листинга.~\ref{malloc:code})
\end{itemize}
Число элементов соответствует размеру сети (см
Листинг.~\ref{network:size:code}).

\clearpage
\section{Расшифровка отладочной печати}
Результат работы функции \verb'print_neuron' выглядит так:

\begin{lstlisting}[
  caption={Пример отладочной печати},
  captionpos=b,
  label={output:debug}]
 веса                        возраст                   локальная   фактор      группа
                            соединений                ошибка      полезности
w:  0.0  -0.6   0.2  -0.4   a: - 0 2 - - - - - - -   e: 0.100   u: 0.600   g: 0
w: -0.0  -1.7   0.8   0.8   a: 0 - 1 - - - - - - -   e: 0.200   u: 0.700   g: 0
w:  0.3  -0.9   0.2  -1.1   a: 2 1 - - - - - - - -   e: 0.300   u: 0.800   g: 0
w:  0.4  -1.2   0.5  -0.7   a: - - - - 3 - - - - -   e: 0.400   u: 0.900   g: 3
w: -0.5   0.2   0.7   0.5   a: - - - 3 - - 0 - - -   e: 0.500   u: 0.000   g: 3
w: -0.3   2.4   4.4   5.3   a: - - - - - - 0 - - -   e: 1.607   u: 7.700   g: 3
w: -0.4   1.3   2.6   2.9   a: - - - - 0 0 - - - -   e: 1.607   u: 8.850   g: 3
-
w:  0.1  -1.2   3.4   5.6   a: - - - - - - - - - -   e: 1.300   u: 2.400   g: 2
-
\end{lstlisting}

Номер группы используются только для картинки и в расчётах не фигурирует.


Таблица возрастов соединений нейронов (нумерация идёт с \O):
\begin{itemize}
\item число обозначает возраст соединения между нейроном в текущей
  строке и нейроном в столбце
\item --- (прочерк) обозначает отсутствие соединения
  (\verb'NOT_CONNECTED') между нейроном в текущей строке и нейроном в
  столбце
\end{itemize}
\begin{lstlisting}[
  caption={Расшифровка таблицы возрастов нейронов: возраст соединения указан в скобках},
  captionpos=b,
  label={conn:age:decryption:debug}]
     0 1 2 3 4 5 6 7 8 9
0 a: - 0 2 - - - - - - - нейрон №0 соединён с нейронами №1 (0) и №2 (2)
1 a: 0 - 1 - - - - - - - нейрон №1 соединён с нейронами №0 (0) и №2 (1)
2 a: 2 1 - - - - - - - - нейрон №2 соединён с нейронами №0 (2) и №1 (1)
3 a: - - - - 3 - - - - - нейрон №3 соединён с нейроном  №4 (3)
4 a: - - - 3 - - 0 - - - нейрон №4 соединён с нейронами №3 (3) и №6 (0)
5 a: - - - - - - 0 - - - нейрон №5 соединён с нейроном  №6 (0)
6 a: - - - - 0 0 - - - - нейрон №6 соединён с нейронами №4 (0) и №5 (0)
7 -                      нейрон №7 не существует
8 a: - - - - - - - - - - нейрон №8 не соединён ни с одним нейроном
9 -                      нейрон №9 не существует
\end{lstlisting}

Соответствующая этой отладочной печати, нейронная сеть --- будет
выглядеть так:
\begin{figure}[h]
  \center
  \includegraphics[width=5cm]{01.eps} % fixme: strange error with path "../img/01.eps"
  \caption{для Листинга.~\ref{output:debug} и~\ref{conn:age:decryption:debug}}
  \label{graph:img}
\end{figure}

\end{document}
